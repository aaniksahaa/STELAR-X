\documentclass[11pt]{article}
\usepackage{amsmath, amssymb, amsthm}
\usepackage{fullpage}

\title{STELAR: A Statistically Consistent Coalescent-Based Species Tree Estimation Method by Maximizing Triplet Consistency}
\author{Mazharul Islam, Kowshika Sarker, Trisha Das, Rezwana Reaz, and Md. Shamsuzzoha Bayzid}

\begin{document}

\maketitle

\begin{abstract}
Species tree estimation is frequently based on phylogenomic approaches that use multiple genes
from throughout the genome. However, estimating a species tree from a collection of gene trees can be complicated
due to the presence of gene tree incongruence resulting from incomplete lineage sorting (ILS), which is modelled by
the multi-species coalescent process. Maximum likelihood and Bayesian MCMC methods can potentially result in
accurate trees, but they do not scale well to large datasets.

We present STELAR (Species Tree Estimation by maximizing tripLet AgReement), a new fast and highly
accurate statistically consistent coalescent-based method for estimating species trees from a collection of gene trees.
We formalized the constrained triplet consensus (CTC) problem and showed that the solution to the CTC problem is a
statistically consistent estimate of the species tree under the multi-species coalescent (MSC) model. STELAR is an
efficient dynamic programming based solution to the CTC problem which is highly accurate and scalable. 
\end{abstract}

\section{Background}
Estimated species trees are useful in many biological analyses, but accurate estimation of species trees can be quite
complicated. Species tree inference can potentially result in accurate evolutionary history using data from multiple
loci. However, combining multi-locus data is difficult, especially in the presence of gene tree discordance. 
A traditional approach to species tree estimation is concatenation (combined analysis), which concatenates gene
sequence alignments into a supergene matrix, and then estimates the species tree using a sequence-based tree estimation technique. 
Although widely used, concatenation can be statistically inconsistent and return incorrect trees with high confidence.

Recent advances have produced methods that explicitly take gene tree discordance into account under the multi-species coalescent (MSC) model.
Several species-tree estimation methods (ASTRAL, MP-EST, *BEAST, NJst, BUCKy, GLASS, STEM, SNAPP, SVDquartets, STEAC, ASTRID) are
statistically consistent under MSC. However, *BEAST is computationally intensive and impractical for large datasets.
Triplet- and quartet-based methods (ASTRAL, MP-EST, SuperTriplets) are scalable and accurate. 

We present STELAR, a coalescent-based method which finds a species tree that agrees with the largest number of triplets induced by the gene trees.
STELAR is statistically consistent under the MSC, polynomial-time, and highly accurate.

\section{Implementation}
The design of ASTRAL and other quartet-based methods relies on the fact that unrooted 4-taxon species trees do not contain an anomaly zone.
We use a similar design in STELAR, utilizing the fact that rooted 3-taxon species trees do not contain anomaly zone.
STELAR is provably statistically consistent under the MSC model, but the guarantee can fail under gene duplication/loss or recombination.

\section{Definitions and Notation}
Let $T$ be a binary rooted tree, leaf-labelled by species set $X$ with $n$ taxa. Each internal node $u$ defines a subtree-bipartition $SBP_T(u)$. 
Let $L(T)$ denote the leaf set. For $r = \{a,b,c\}\subset X$, possible rooted topologies are $a|bc$, $b|ca$, $c|ab$. 
The dominant topology is the most frequent one under the MSC. 
We denote the set of all triplets induced by $T$ as $TR(T)$.

\section{Problem Definition}
The Constrained Triplet Consensus (CTC) problem is defined as:

\begin{itemize}
\item \textbf{Input:} a set $G$ of rooted gene trees on $X$ and a set $SBP$ of subtree-bipartitions.
\item \textbf{Output:} a species tree $ST$ on $X$ that maximizes triplet agreement with $G$ and has all its subtree-bipartitions in $SBP$.
\end{itemize}

\textbf{Theorem 1.} Given a set $G$ of true gene trees, the solution to the CTC problem (both exact and constrained versions) is a statistically consistent estimator of the species tree topology under MSC.

\section{Algorithmic Design of STELAR}
STELAR uses a dynamic programming (DP) solution to CTC. The key is to compute the number of triplets in gene trees that agree with a given subtree in the species tree.

\textbf{Lemma 1.} The number of triplets mapped to a subtree-bipartition $x=(X_1|X_2)$ where $|X_1|=n_1$, $|X_2|=n_2$ is:
\[
N_T(n_1,n_2) = \binom{n_1}{2}\binom{n_2}{1} + \binom{n_2}{2}\binom{n_1}{1} = \frac{n_1 n_2 (n_1+n_2-2)}{2}.
\]

\textbf{Lemma 2.} For two subtree-bipartitions $x=(X_1|X_2)$ and $y=(Y_1|Y_2)$, the number of triplets mapped to both is:
\[
M(x,y) = N_T(|X_1\cap Y_1|,|X_2\cap Y_2|) + N_T(|X_1\cap Y_2|,|X_2\cap Y_1|).
\]

The triplet consistency score for subtree-bipartition $x$ with respect to gene trees $G$ is:
\[
TC_G(x) = \sum_{gt \in G} \sum_{y \in SBP(gt)} M(x,y).
\]

Thus, the total score for species tree $ST$ is:
\[
TC_G(ST) = \sum_{x \in SBP(ST)} TC_G(x).
\]

\section{Dynamic Programming}
We compute $V(A)$, the score of an optimal subtree on leaf set $A \subseteq X$.

\begin{itemize}
\item Base case: $|A|=1 \implies V(A)=0$.
\item Recursive relation: 
\[
V(A) = \max_{A'|A-A'\in SBP} \big( V(A') + V(A-A') + TC_G(A'|A-A') \big).
\]
\end{itemize}

\section{Running Time Analysis}
\textbf{Lemma 3.} Given $k$ gene trees on $n$ taxa, computing $TC_G(x)$ takes $O(n^2k)$ time.

\textbf{Theorem 2.} For $k$ gene trees on $n$ taxa and $SBP$ subtree-bipartitions, STELAR runs in $O(n^2 k |SBP|^2)$ time.

\end{document}
